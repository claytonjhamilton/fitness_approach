% !TEX TS-program = xelatex
% !TEX encoding = UTF-8 Unicode

%%%%%%%%%%%%%%%%%%%%%%%%%%%%%%%%%%%%%%%%%
% Physical Self-Care Routine 
% Clayton J. Hamilton, PharmD
%%%%%%%%%%%%%%%%%%%%%%%%%%%%%%%%%%%%%%%%%

\documentclass[12pt, letterpaper]{article}
\usepackage[table,xcdraw]{xcolor}
% \usepackage[style=numeric,citestyle=apa]{biblatex} % Use citation manager
\usepackage[backend=biber,style=numeric,sortcites,sorting=nty,backref,natbib,hyperref]{biblatex}
\addbibresource{citations.bib} % Place formatted citations in this file using https://bioinformatics.org/texmed/
\usepackage[T1]{fontenc}
\usepackage[left=1.9cm, top=1.9cm, right=1.9cm, bottom=1.9cm, footskip=.5cm]{geometry} % Configure page margins with geometry

\title{Physical Self-Care Routine}
\author{Clayton J. Hamilton, PharmD}
\date{}

\renewcommand*\contentsname{Summary}

%%%%%%%%%%%%%%%%%%%%%%%%%%%%%%%%%%%%%%%%%
% Cover page
\begin{document}

\maketitle

\centerline{\textit{A structured approach to improving general fitness and reducing injury risk}}

\tableofcontents

\newpage % Force a new page for looks

%%%%%%%%%%%%%%%%%%%%%%%%%%%%%%%%%%%%%%%%%

\section{Overview}

Intermediate climbing ability is loosely defined as bouldering V4-V8, top rope 5.11 to 5.12, 
and actively climbing longer than a year.

\begin{enumerate}
      \item Climbing and climbing-specific exercises 3-4x per week
      \item Stability/Antagonist training 2x per week
      \item Aerobic exercise 1-3x per week
  \end{enumerate}

% Please add the following required packages to your document preamble:
% \usepackage[table,xcdraw]{xcolor}
% If you use beamer only pass "xcolor=table" option, i.e. \documentclass[xcolor=table]{beamer}
\begin{table}[h]
      \begin{tabular}{lllllll}
      \hline
      \multicolumn{7}{|l|}{\textbf{Climbing Routine \#1}} \\ \hline
      \rowcolor[HTML]{C0C0C0} 
      \multicolumn{1}{|l|}{\cellcolor[HTML]{C0C0C0}Monday} &
        \multicolumn{1}{l|}{\cellcolor[HTML]{C0C0C0}Tuesday} &
        \multicolumn{1}{l|}{\cellcolor[HTML]{C0C0C0}Wednesday} &
        \multicolumn{1}{l|}{\cellcolor[HTML]{C0C0C0}Thursday} &
        \multicolumn{1}{l|}{\cellcolor[HTML]{C0C0C0}Friday} &
        \multicolumn{1}{l|}{\cellcolor[HTML]{C0C0C0}Saturday} &
        \multicolumn{1}{l|}{\cellcolor[HTML]{C0C0C0}Sunday} \\ \hline
      \multicolumn{1}{|l|}{C/SMR} &
        \multicolumn{1}{l|}{RC/AE/SMR} &
        \multicolumn{1}{l|}{RC/AE/SMR} &
        \multicolumn{1}{l|}{UB/AE/SMR} &
        \multicolumn{1}{l|}{C/SMR} &
        \multicolumn{1}{l|}{RC} &
        \multicolumn{1}{l|}{UB/AE} \\ \hline
      \multicolumn{7}{|l|}{\textbf{Climbing Routine \#2}} \\ \hline
      \rowcolor[HTML]{C0C0C0} 
      \multicolumn{1}{|l|}{\cellcolor[HTML]{C0C0C0}Monday} &
        \multicolumn{1}{l|}{\cellcolor[HTML]{C0C0C0}Tuesday} &
        \multicolumn{1}{l|}{\cellcolor[HTML]{C0C0C0}Wednesday} &
        \multicolumn{1}{l|}{\cellcolor[HTML]{C0C0C0}Thursday} &
        \multicolumn{1}{l|}{\cellcolor[HTML]{C0C0C0}Friday} &
        \multicolumn{1}{l|}{\cellcolor[HTML]{C0C0C0}Saturday} &
        \multicolumn{1}{l|}{\cellcolor[HTML]{C0C0C0}Sunday} \\ \hline
      \multicolumn{1}{|l|}{RC/AE/SMR} &
        \multicolumn{1}{l|}{RC/AE/SMR} &
        \multicolumn{1}{l|}{UB/AE/SMR} &
        \multicolumn{1}{l|}{C/SMR} &
        \multicolumn{1}{l|}{RC/SMR} &
        \multicolumn{1}{l|}{UB/AE} &
        \multicolumn{1}{l|}{C/AE} \\ \hline
      \multicolumn{7}{l}{\begin{tabular}[c]{@{}l@{}}SMR = Self-Myofascial Release\\ UB = Upper Body\\ AE = Aerobic Exercise\\ C = Core\\ RC = Rock Climbing\end{tabular}}
      \end{tabular}
      \end{table}

\newpage % Force a new page for looks

%%%%%%%%%%%%%%%%%%%%%%%%%%%%%%%%%%%%%%%%%

\section{Self-Myofascial Release (SMR) - Once Daily}

A systematic review found that SMR may reduce perceived pain after 
intense exercise and improve joint range of motion.\cite{pmid26618062} Further, a small study found that
 emphasizing static pressure on myofascial trigger points during each session may help resolve muscle impairments.\cite{pmid30765920} 
 \\
 \\
Time to complete: ~9 minutes

\subsection{Foam Roller}

For each of the following perform 10 complete rolls (1 roll = top to bottom to top). Emphasize static pressure on trigger points.
\begin{enumerate}
    \item Upper Back
    \item Latissimus Dorsi
    \item Gluteus Maximus \\ Place ankle on knee; roll one side at a time
    \item Hip Flexors
    \item IT band
\end{enumerate}

\subsection{Ball}
Place ball on floor and roll out the following with 10 complete rolls per side. Emphasize static pressure on trigger points.
\begin{enumerate}
    \item Pectoral Release \\ Lie on stomach, cross one arm behind back, and roll armpit to sternum on that side
    \item Scapular Release \\ Lie on back, cross arms over chest, and roll along medial border of scapula
\end{enumerate}

\newpage % Force a new page for looks

%%%%%%%%%%%%%%%%%%%%%%%%%%%%%%%%%%%%%%%%%

\section{Strength Training}

This section emphasizes strength training in relation to rock climbing.
The intention is to enhance strength and reduce injury risk. Larger number
of reps per set increase muscle fiber endurance while limiting excess 
muscle bulk that is unnecessary for climbing. Aim for 48 hours of rest 
between sessions to allow for full recovery before training the same muscle
 group again. Reduce injury risk by taking at least 3 days off from climbing 
 (specifically) per week.

\subsection{Upper Body - 2-3x weekly}

Consider doing one set of these with light weight pre-climbing workout then two sets with higher resistance post-workout.

\subsubsection{Wrist Stabilizers}

\begin{enumerate}
    \item Reverse Wrist Curls with dumbbell \\ 
          Two sets: First is 20-25 reps with 3 minute rest then heavier weight 10-15 reps.
    \item Wrist Extension Isometric \\ 
          Rest forearm on flat surface and hold dumbbell straight out over space. Hold for 
          45-60 seconds per side with 3 minute rest in between. Long term, aim for 2 minute hold.
    \item Wide Pinch with Wrist Extension \\ 
          Find a wide grip device, such as 2-3 2x4s screwed together, that weight can be hung 
          off of. Standing tall, grip the block and weight with arm by side. Hold for 
          10-30 seconds with lighter weight at first. Eventually graduate heavier weights 
          with 10 second hold times. Perform three sets of three reps per hand consecutively 
          with 3 minute rest between sets.
    \item Pronator Isolation \\ 
          Sit, palm up, elbow on knee. Hold a 3 lb sledgehammer upright in one hand and raise and 
          lower to outside of body. 20-25 reps with 2 sets each hand.
    \item Reverse Arm Curls \\ 
          Using weighted bent barbell with hands on top of bar while standing, perform 15-20 reps 
          with light weight then 2 more sets with heavier weight.
\end{enumerate}

\subsubsection{Rotator Cuff and Scapular Stability}

\begin{enumerate}
    \item Elastic band or dumbbell internal rotation \\ 
          Only move hand and forearm across the body. Recommended to lie on side with arm on floor 
          at 90 degrees. Lift dumbbell from floor to opposite side of body. 
          Perform 2 sets of 25 reps on each side.
    \item Elastic band or dumbbell external rotation \\ 
          Same as above, only moving in opposite direction. Dumbbell is moved by upper arm if 
          lying on floor.
    \item T exercise \\ 
          Lay prone on bench with arms hanging down. Raise arms straight out and squeeze 
          shoulder blades together. 2 sets of 10-20 reps with 3 minute rest in between.
    \item Y exercise \\ 
          Lay prone on bench with arms hanging down. Raise arms up into a "Y" shape with 
          hands just beyond shoulder width apart out and squeeze shoulder blades down. 
          2 sets of 10-20 reps with 3 minute rest in between.
    \item Scapular Push-up \\ 
          Start in push-up position. Allow chest to sag down then drive into the top position. 
          Range may only be several inches. 2 sets of 10-20 reps.
    \item Scapular Pull-up \\ 
          Hanging from pull-up bar, draw shoulder blades down and together in a reverse shrug. 
          Head shifts back and chest raises upward. Only 1 or 2 sets at most.
\end{enumerate}

\subsubsection{Antagonist Muscle Group Training}

\begin{enumerate}
    \item Shoulder Press \\ 
          2 sets of 15-20 reps.
    \item Bench Press \\ 
          For total dumbbell weight, start at 30\% of bodyweight and work towards 75\% of BW. 2 sets of 15-20 reps. 
          Can substitute with pushups.
    \item Dips \\ 
          Rings are ideal to increase stability. Never lower beyond 90 degrees. 2-3 sets of 8-20 reps with 2 seconds per rep.
\end{enumerate}

\subsection{Core - 3+ days per week}

Pick 2-3 exercises from each of the following groups.

\subsubsection{Anterior Core}

\begin{enumerate}
    \item Feet-Up Crunches \\ 
          Lie on back with legs bent and shin parallel with floor (feet in air). Perform as many crunches as possible, 
          rest 3 minutes, then do a second set. Long-term goal is 50-100 crunches.
    \item Hanging Knee Lifts \\ 
          Hang from pull-up bar and bring knees to chest then back down. Perform as many as possible, rest 3 minutes, 
          then do a second set. Long-term goal is 15-20 reps.
    \item Dip-bar leg raises \\
          Starting on dip bar, raise legs until parallel with floor. Lower to a 45 degree angle then lift again to 
          parallel. Increase difficulty by holding a small dumbbell between feet. 
      \item V-ups\\
         Lie flat on back with arms overhead. Spike hands and feet over belly at the same time then relax back down.
         Don't allow hands to touch ground.
      \item Plank \\
          1 minute forearm plank, 2x 30 second one-arm planks.
    \item Elevated Plank \\
          Standard high plank with feet on exercise ball. 3 rounds, 60 seconds each, and 60 second rest between.
    \item Mountain Climber Plank \\ 
          From plank position, maintain a perfectly straight back, and bring knee out and towards elbow. 
          Perform 20-50 slow repetitions (3 sec per rep). 
          Should take 1-2 minutes to complete a set.
    \item One-Arm Elbow and Side Plank \\ 
          From plank position, turn body and reach one arm vertically and hold for 2 seconds. 1 rep is both sides. 
          10-20 reps at about 6 seconds per rep. 
    \item "Marine Core" (difficult) \\ 
          Hang TRX sling with handles at waist height when kneeling on floor. Grip handles from kneeling position 
          and extend arms while bending forward and maintaining a straight spine and hold 2 seconds. 2-3 sets of 5-12 reps.
    \item Windshield Wipers (difficult) \\ 
          Hang from pull-up bar with palms away from head. Lift shins to bar then lower legs to one side, return to bar, 
          then lower the other way (1 rep). 2-3 sets of 6-12 reps w/ 3 minute rest in between.    
\end{enumerate}

\subsubsection{Posterior Core}

\begin{enumerate}

    \item Superman\\ 
          Lie on stomach and extend arms forward. Raise one arm and opposite leg as high as is comfortable for 1 second.
          Lower and repeat on other side for 1 rep. 2-3 sets of 20 reps w/ 3 minute rest between.
    \item Reverse Sit Up\\ 
          Perform 10-15 reps for 2-3 sets w/ 3 min rest between.
    \item Reverse Mountain Climber Plank\\ 
          Assume reverse plank position and lift knee out and away from other leg. 8-15 reps, 2-3 sets, 3 min rest between.
    \item Side Hip Raises\\ 
          Lie on side with only one hand and foot on floor. Keep supporting arm and spin straight and lower hips to touch floor
          then return to starting position. 2 sets of 10-20 reps per side w/ 3 min rest between.
\end{enumerate}

\subsubsection{Total Core and Posterior Chain}

\begin{enumerate}

    \item Dumbbell Snatch\\ 
          Stand with feet shoulder-width and toes pointed out 20 degrees. With dumbbell between feet, bend equally between knees and hips. 
          Explode upwards and shrug shoulders when at point of standing, allowing dumbbell to launch upwards and overhead. 
          Hold 1 second then return dumbbell to floor. Start w/ 15-25 lbs and increase. 5-8 reps then repeat w/ other hand. 
          when comfortable with technique.
    \item Sumo Deadlift with Dumbbell/Kettlebell\\ 
          Spread feet 1.5 shoulder widths apart (wide) with toes pointed out 20 degrees. Grasp dumbbell on both sides. 
          Straighten knees, which should reach full extension before hips do. At the top, drive hips forward and pull
          shoulders back. Start with 15-30 lbs and favor good form over heavy weight. 
          10-15 reps and rest 3 minutes before optional second set.
    \item Barbell Deadlift \\ 
    \item Barbell Squat \\ 
    
\end{enumerate}

\subsubsection{Climbing-Specific Core}

\begin{enumerate}

    \item Roof Lever-ups \\ 
          Straight-arm hang from holds on bouldering cave roof. Pull up halfway with arms (arms at 90 degrees) then drop head/shoulders
          back and lift feet to latch a foothold as far away as possible. Match feet and slowly relax core as much as possible without
          losing the foot hold. Learn the amount of friction required to stay on. Release feet and do again. 
          2-3 sets of 3-8 reps w/ 3 minute rest between.
    \item Steep Wall Cut and Catch \\ 
          In bolder cave, choose an easy-moderate problem. Cut feet at every hand move then immediately bring feet back to the wall. 
          3-5 laps with 1-3 minutes rest between.
    \item Steep Wall Traversing \\ 
          Traverse across a wall overhanging 30-50 degrees using large handholds and small footholds for 60-120 seconds. 
          Lead with feet by reaching/swinging foot out as far as possible and stiffen core.
          Rest 3 minutes then perform a second set in the opposite direction.
    \item Front Lever (hard) \\ 
          Pull-up bar halfway, extend head backwards and push hands forward until entire body is parallel with floor. 
          Hold 2 seconds then slowly lower to start. 2-3 sets of 2-5 reps  w/ 3-5 minute rest between sets.
    
\end{enumerate}

\subsection{Finger and Lock-off Strength}

\subsubsection{Finger Strength}

\begin{enumerate}
      
       \item Minimum Edge Fingerboard \\ 
            Do a series of 5 12-second hangs with exactly 3 minutes rest in between. 
            Each hang should be near max (exertion of 9-9.5 out of 10). Begin by 
            doing just two sets and work your way up to 5. Be sure to use open or half crimp. 

       \item Hand Play \\ 
            Find a fingerboard or campus board with a support for your feet in a place to 
            relieve some of your weight. Moving from hand to hand, shaking out when needed 
            say on the board for 3-5 minutes shifting your weight. Keep the pump under
            control, and step off before the pump becomes deep or painful. Err on the side
            of too little pump. 4-6 sets with nearly equal resting to Climbing.

\end{enumerate}

\newpage % Force a new page for looks

%%%%%%%%%%%%%%%%%%%%%%%%%%%%%%%%%%%%%%%%%

\section{Maximum Strength Testing and Training}

\subsection{Testing}
Perform this series of tests once per month or every 4 months to track progression
of pull muscle strength.

\subsubsection{1. Max Weight Pull-up Test}

Starting with XX lbs hanging from harness, complete a full pull-up.
Keep adding increments of 10 lbs until unable to complete. 

\subsubsection{2. Max Weight Five Pull-up Test}

What is the maximum amount of weight hanging from belay loop that still
allows you to complete 5 pull-ups?

\subsubsection{3. Max pull ups}

How many full pull-ups (straight arm) can be completed? 

\subsection{Pull Strength Training}

\subsubsection{Weighted pull-ups}

After being able to do at least 8 bodyweight pull-ups, add 10-20 lbs hanging 
from belay loop to start. Explode up and slowly lower. Do not allow yourself to hang straight-armed. 
3-6 sets of 5 pull-ups. Aim for up to 50\% of bodyweight. 

\subsubsection{Square pull-ups}

Grip 50\% wider than shoulders. 2 sets of 4-6 pull-ups alternating direction each set.

\subsubsection{One-arm lock-offs}

The gateway to the one-arm pull-up. Perform a pull-up to a high lock-off with one arm. Cheek pressed into the bar. 
Immediately release other hand. Lower slowly as you lose the position. Once your arm hits 120 degrees, immediately
 perform a full pull-up with both arms and perform the lock-off with the other arm. 
  1-2 sets of 2 reps per side.

\newpage % Force a new page for looks

%%%%%%%%%%%%%%%%%%%%%%%%%%%%%%%%%%%%%%%%%

\section{Climbing}

\item World Cup Simulator \\ 
Pick a difficult project and work on it for 15 minutes. When you fail, start from 
point of failure and keep working upwards. After 15 minutes is up, climb every route
 in the gym of a specific grade in 15 minutes (i.e. if your project is V4 then climb 
 every V1 in the gym). Long term aim for 6 sets of these 30 minute rounds. If you send the
 project early, pick another.

\item Route Climbing Intervals \\ 
Choose routes that are 2-4 number grades below your limit (for us 5.8-10). 
Rest while belaying loved one. Work up to climbing 20 routes. If you develope 
a deep muscular pump or labored breathing, immediately move on to easier routes.

\item Bouldering \\ 
Find an overhanging bouldering route and climb 3x (1 set) w/ 2-3 minute rest
in between. Find another problem that targets a different grip position
(crimp-only, pocket-only, pinch-only) and climb another set. 
Continue this way for 30-60 minutes (3-10 total sets).
Once complete, complete a finger isolation exercise.
NOTE: a weight belt (5-20 lbs) can be used for advanced climbers. 
Use less difficult grades and avoid crimps. Use extra caution to avoid injury.

\item Bouldering 4x4s \\ 
Climb first bouldering problem which should not take longer than 
30 seconds. Rest 30 seconds. Complete 4 laps on one problem (1 set)
while alternating climbing/resting. Switch timing laps with your partner
which allows 4 minuts of rest in between sets. Complete 4 sets which
should take ~30 minutes.
%%%%%%%%%%%%%%%%%%%%%%%%%%%%%%%%%%%%%%%%%

% \section{Yoga - 3x weekly(?)}

% \newpage % Force a new page for looks

%%%%%%%%%%%%%%%%%%%%%%%%%%%%%%%%%%%%%%%%%

% \section{Aerobic Training - 3x weekly(?)}

% \newpage % Force a new page for looks

%%%%%%%%%%%%%%%%%%%%%%%%%%%%%%%%%%%%%%%%%
\section{References}
\printbibliography[heading=none] 
%%%%%%%%%%%%%%%%%%%%%%%%%%%%%%%%%%%%%%%%%

\end{document}