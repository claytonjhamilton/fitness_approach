% !TEX TS-program = xelatex
% !TEX encoding = UTF-8 Unicode

%%%%%%%%%%%%%%%%%%%%%%%%%%%%%%%%%%%%%%%%%
% Physical Self-Care Routine 
%
% Original author:
% Clayton J. Hamilton, PharmD
%%%%%%%%%%%%%%%%%%%%%%%%%%%%%%%%%%%%%%%%%

\documentclass[12pt, letterpaper]{article}
% \usepackage[style=numeric,citestyle=apa]{biblatex} % Use citation manager
\usepackage[backend=biber,style=numeric,sortcites,sorting=nty,backref,natbib,hyperref]{biblatex}
\addbibresource{citations.bib} % Place formatted citations in this file using https://bioinformatics.org/texmed/
\usepackage[T1]{fontenc}
\usepackage[left=1.9cm, top=1.9cm, right=1.9cm, bottom=1.9cm, footskip=.5cm]{geometry} % Configure page margins with geometry

\title{Physical Self-Care Routine}
\author{Clayton J. Hamilton, PharmD}
\date{}

\renewcommand*\contentsname{Summary}

%%%%%%%%%%%%%%%%%%%%%%%%%%%%%%%%%%%%%%%%%
% Cover page
\begin{document}

\maketitle

\centerline{\textit{A structured approach to improving general fitness and reducing injury risk}}

\tableofcontents

\newpage % Force a new page for looks

%%%%%%%%%%%%%%%%%%%%%%%%%%%%%%%%%%%%%%%%%

\section{Self-Myofascial Release (SMR) - Once Daily}

A systematic review found that SMR may reduce perceived pain after 
intense exercise and improve joint range of motion.\cite{pmid26618062} Further, a small study found that
 emphasizing static pressure on myofascial trigger points during each session may help resolve muscle impairments.\cite{pmid30765920} 
 \\
 \\
Time to complete: ~9 minutes

\subsection{Foam Roller}

For each of the following perform 10 complete rolls (1 roll = top to bottom to top). Emphasize static pressure on trigger points.
\begin{enumerate}
    \item Upper Back
    \item Latissimus Dorsi
    \item Gluteus Maximus \\ Place ankle on knee; roll one side at a time
    \item Hip Flexors
    \item IT band
\end{enumerate}

\subsection{Ball}
Place ball on floor and roll out the following with 10 complete rolls per side. Emphasize static pressure on trigger points.
\begin{enumerate}
    \item Pectoral Release \\ Lie on stomach, cross one arm behind back, and roll armpit to sternum on that side
    \item Scapular Release \\ Lie on back, cross arms over chest, and roll along medial border of scapula
\end{enumerate}

\newpage % Force a new page for looks

%%%%%%%%%%%%%%%%%%%%%%%%%%%%%%%%%%%%%%%%%

\section{Strength Training}

This section emphasizes strength training in relation to rock climbing.
The intention is to enhance strength and reduce injury risk. 

\subsection{Upper Body - 2x weekly}
Consider doing one set of these with light weight pre-climbing workout then two sets with higher resistance post-workout.

\subsubsection{Wrist Stabilizers}

\begin{enumerate}
    \item Reverse Wrist Curls with dumbbell \\ Two sets: First is 20-25 reps with 3 minute rest then heavier weight 10-15 reps.
    \item Wrist Extension Isometric \\ Rest forearm on flat surface and hold dumbbell straight out over space. Hold for 45-60 seconds per side with 3 minute rest in between. Long term, aim for 2 minute hold.
    \item Wide Pinch with Wrist Extension \\ Find a wide grip device, such as 2-3 2x4s screwed together, that weight can be hung off of. Standing tall, grip the block and weight with arm by side. Hold for 10-30 seconds with lighter weight at first. Eventually graduate heavier weights with 10 second hold times. Perform three sets of three reps per hand consecutively with 3 minute rest between sets.
    \item Pronator Isolation \\ Sit, palm up, elbow on knee. Hold a 3 lb sledgehammer upright in one hand and raise and lower to outside of body. 20-25 reps with 2 sets each hand.
    \item Reverse Arm Curls \\ Using weighted bent barbell with hands on top of bar while standing, perform 15-20 reps with light weight then 2 more sets with heavier weight.
\end{enumerate}

\subsubsection{Rotator Cuff and Scapular Stability}

\begin{enumerate}
    \item Elastic band or dumbbell internal rotation \\ Only move hand and forearm across the body. Recommended to lie on side with arm on floor at 90 degrees. Lift dumbbell from floor to opposite side of body. Perform 2 sets of 25 reps on each side.
    \item Elastic band or dumbbell external rotation \\ Same as above, only moving in opposite direction. Dumbbell is moved by upper arm if lying on floor.
    \item T exercise \\ Lay prone on bench with arms hanging down. Raise arms straight out and squeeze shoulder blades together. 2 sets of 10-20 reps with 3 minute rest in between.
    \item Y exercise \\ Lay prone on bench with arms hanging down. Raise arms up into a "Y" shape with hands just beyond shoulder width apart out and squeeze shoulder blades down. 2 sets of 10-20 reps with 3 minute rest in between.
    \item Scapular Push-up \\ Start in push-up position. Allow chest to sag down then drive into the top position. Range may only be several inches. 2 sets of 10-20 reps.
    \item Scapular Pull-up \\ Hanging from pull-up bar, draw shoulder blades down and together in a reverse shrug. Head shifts back and chest raises upward. Only 1 or 2 sets at most.
\end{enumerate}

\subsubsection{Antagonist Muscle Group Training}

\begin{enumerate}
    \item Shoulder Press \\ 2 sets of 15-20 reps.
    \item Bench Press or Push-ups \\ Start at 30\% of bodyweight and work towards 75\% of BW. 2 sets of 15-20 reps.
    \item Dips \\ Rings are ideal to increase stability. Never lower beyond 90 degrees. 2-3 sets of 8-20 reps with 2 seconds per rep.
\end{enumerate}
    

\newpage % Force a new page for looks

%%%%%%%%%%%%%%%%%%%%%%%%%%%%%%%%%%%%%%%%%

\section{Yoga - 3x weekly(?)}

\newpage % Force a new page for looks

%%%%%%%%%%%%%%%%%%%%%%%%%%%%%%%%%%%%%%%%%

\section{Aerobic Training - 3x weekly(?)}

\newpage % Force a new page for looks

%%%%%%%%%%%%%%%%%%%%%%%%%%%%%%%%%%%%%%%%%

\section{Rock Climbing Training - 3x weekly(?)}

\newpage % Force a new page for looks

%%%%%%%%%%%%%%%%%%%%%%%%%%%%%%%%%%%%%%%%%
\section{References}
\printbibliography[heading=none] 
%%%%%%%%%%%%%%%%%%%%%%%%%%%%%%%%%%%%%%%%%

\end{document}